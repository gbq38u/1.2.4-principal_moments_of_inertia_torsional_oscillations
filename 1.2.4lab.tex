\documentclass[a4paper,10pt]{article} 
% --- Пакеты и кодировки ---
\usepackage[left=3cm,right=3cm]{geometry}   % поля
\usepackage{float}                           % [H] для рисунков

\usepackage[utf8]{inputenc}                  % кириллица в исходнике (на новом LaTeX можно опустить)
\usepackage[T2A]{fontenc}                    % шрифтовая кодировка
\usepackage[russian]{babel}                  % локализация/переносы
\usepackage{csquotes}

% --- Графика (ВАЖНО: без dvips!) ---
\usepackage{graphicx}                        % Должен идти до \graphicspath
\graphicspath{{./}{pictures/}}               % ищем картинки и в текущей папке, и в pictures/
\DeclareGraphicsExtensions{.pdf,.png,.jpg}

% --- Математика ---
\usepackage{amsmath,amsfonts,amssymb,amsthm,mathtools}

% --- Ссылки/закладки ---
\usepackage{hyperref}                        % обычно подключают перед bookmark
\usepackage{bookmark}

% --- Границы плавающих объектов ---
\usepackage[section]{placeins}               % для \FloatBarrier

% --- Титул ---
\title{Отчёт о выполнении лабораторной работы\\
\textbf{Определение главных моментов инерции твердых тел с помощью крутильных колебаний}\\
\textbf{Лабораторная работа 1.2.4}}
\author{Г.А.Кузин }
\date{ Группа Б01-501 г. Долгопрудный, ФРКТ МФТИ, 23.11.2025 \\}

\begin{document}
\maketitle
\newpage
\section*{Аннотация}
Экспериментально исследован эллипсоид инерции твёрдого тела методом крутильных колебаний. Измерены периоды колебаний прямоугольного параллелепипеда, закреплённого в специальной рамке, относительно главных осей и осей, проходящих по диагоналям тела. По результатам измерений определены моменты инерции относительно различных осей, проверены соотношения, связывающие периоды колебаний с размерами параллелепипеда, а также установлено совпадение периодов для куба. На основе полученных данных построены сечения эллипсоида инерции и показано, что экспериментально найденная форма эллипсоида согласуется с теоретическими выражениями для моментов инерции и уравнением эллипсоида инерции.\newpage
\section*{Введение}
Во многих современных приборах точность и надёжность работы напрямую связаны с правильным учётом инерционных свойств их корпусов и узлов. Характерный пример – гиростабилизированная платформа с телевизионной и тепловизионной аппаратурой беспилотного летательного аппарата. Блок оптико-электронного оборудования заключён в прямоугольный корпус, внутри которого размещены объективы, матрицы, блоки питания, кабельные трассы. Фактическое распределение массы такого блока значительно отличается от простых расчётных моделей, поэтому моментами инерции часто пренебрегают или задают их приближённо. В результате при резких манёврах БПЛА и работе приводов платформы возникают нежелательные перекрёстные колебания и резонансы, ухудшающие точность наведения и уменьшающие ресурс механики.

Для построения корректной системы управления платформой, выбора жёсткости подвеса и настройки демпфирующих устройств необходимо знать не только значения моментов инерции блока, но и ориентацию его главных осей относительно конструктивных осей подвеса. Прямые расчёты по чертежам оказываются недостаточно точными из-за сложного и частично неизвестного распределения массы внутри корпуса. Поэтому требуется экспериментальный метод определения эллипсоида инерции твёрдого тела, позволяющий по измеряемым периодам крутильных колебаний восстановить моменты инерции относительно различных осей и найти главные оси.

В данной работе на модели в виде прямоугольного параллелепипеда изучается именно такой метод. Освоение способа экспериментального построения эллипсоида инерции позволяет в дальнейшем применять его к реальным приборам типа описанного оптико-электронного блока: измерять их инерционные характеристики после окончательной сборки и тем самым повышать точность расчёта динамики и качество стабилизации.
\newpage
\section*{Методика эксперимента}
В эксперименте система «рамка + исследуемое тело» приводится к
крутильным колебаниям, а основными измеряемыми величинами являются
периоды колебаний при различных ориентациях тела в рамке. Для малых
углов поворота момент инерции тела относительно оси кручения
выражается через период крутильных колебаний системы
\begin{equation}
    I = \frac{f}{4\pi^2}\bigl(T^2 - T_0^2\bigr),
    \label{eq:I_T}
\end{equation}
где $T$ — период колебаний рамки с телом, $T_0$ — период пустой рамки,
$f$ — модуль кручения подвесной проволоки~\cite{Sivukhin,Strelkov}.
Эта формула используется для вычисления экспериментальных значений
моментов инерции всех рассматриваемых тел (параллелепипеда, цилиндра,
куба) относительно заданных осей.

Для прямоугольного параллелепипеда с размерами $a$, $b$, $c$ вдоль осей
$Ox$, $Oy$, $Oz$ измеряются периоды колебаний $T_x$, $T_y$, $T_z$ при
совпадении оси кручения с соответствующими главными осями инерции, а
также период $T_A$ при вращении вокруг диагонали параллелепипеда.
Проверяется соотношение
\begin{equation}
    (a^2 + b^2 + c^2)T_A^2 =
    a^2 T_x^2 + b^2 T_y^2 + c^2 T_z^2,
    \label{eq:TA_rel}
\end{equation}
которое следует из уравнения эллипсоида инерции для данного тела
\cite{Sivukhin,Strelkov}. Аналогично для осей, соединяющих середины
параллельных рёбер (оси $EE'$, $PP'$, $MM'$), по измеренным периодам
$T_E$, $T_P$, $T_M$ проверяются зависимости
\begin{equation}
    (b^2 + c^2)T_E^2 = b^2T_y^2 + c^2T_z^2,
    \label{eq:TE_rel}
\end{equation}
\begin{equation}
    (a^2 + c^2)T_P^2 = a^2T_x^2 + c^2T_z^2,
    \label{eq:TP_rel}
\end{equation}
\begin{equation}
    (a^2 + b^2)T_M^2 = a^2T_x^2 + b^2T_y^2.
    \label{eq:TM_rel}
\end{equation}
Сравнение левой и правой частей формул
\eqref{eq:TA_rel}–\eqref{eq:TM_rel} позволяет экспериментально
подтвердить эллипсоидальный закон изменения момента инерции при
повороте оси.

Для сплошного цилиндра массы $m$, радиуса $R$ и высоты $h$ теоретические
моменты инерции относительно оси симметрии ($Oz$) и диаметра ($Ox$)
равны
\begin{equation}
    I_z^{\text{(th)}} = \frac12 mR^2,\qquad
    I_x^{\text{(th)}} = \frac14 mR^2 + \frac1{12}mh^2.
    \label{eq:I_cyl}
\end{equation}
Экспериментальные значения $I_z^{\text{(exp)}}$ и $I_x^{\text{(exp)}}$
вычисляются по формуле~\eqref{eq:I_T} из периодов $T_z$ и $T_x$; дополнительно
можно сравнить отношение
\begin{equation}
    \frac{T_z^2}{T_x^2} =
    \frac{I_z^{\text{(exp)}}}{I_x^{\text{(exp)}}}
    \approx
    \frac{I_z^{\text{(th)}}}{I_x^{\text{(th)}}},
    \label{eq:ratio_cyl}
\end{equation}
что служит проверкой теоретических выражений для моментов инерции
цилиндра~\cite{Sivukhin}.

В заключительной части работы исследуется куб с ребром $a$, для
которого $a = b = c$. В этом случае из соотношений
\eqref{eq:TA_rel}–\eqref{eq:TM_rel} следует совпадение периодов
крутильных колебаний относительно любой оси, проходящей через центр
массы:
\begin{equation}
    T_x \approx T_y \approx T_z \approx
    T_A \approx T_E \approx T_P \approx T_M.
    \label{eq:cube}
\end{equation}
Измеряя перечисленные периоды и используя формулу~\eqref{eq:I_T}, можно
убедиться, что эллипсоид инерции куба в пределах экспериментальной
точности вырождается в сферу~\cite{Strelkov}.

\newpage
\section*{Результаты и их обсуждение}
В данной работе используется установка для измерения крутильных колебаний,
приведённая на рис.~3. Рамка жёстко соединена с вертикально закреплённой
проволокой, концы которой зажаты в специальных держателях, позволяющих
сообщить системе начальное закручивание и тем самым возбудить крутильные
колебания вокруг вертикальной оси.  

По результатам измерений времени колебаний пустой рамки (табл.~\ref{tab:frame})
найден её собственный период $T_0 = 2{,}57$~с. Это значение далее использовалось
во всех расчётах для учёта собственного момента инерции рамки согласно
формуле для периода крутильных колебаний из раздела «Методика»
(см. формулу~\eqref{eq:I_T}).

\subsection*{Цилиндр}

По данным штангенциркуля (табл.~\ref{tab:cyl-geom}) получены средние значения
размеров цилиндра: высота $h = 49{,}3$~мм и диаметр основания
$d = 88{,}1$~мм (радиус $R = 44{,}1$~мм). Масса цилиндра, измеренная на весах,
составила $m_\text{цил} = 2{,}264$~кг. Теоретические значения моментов инерции
сплошного цилиндра относительно оси симметрии $Oz$ и диаметра $Ox$ были
вычислены по стандартным формулам для однородного цилиндра
(формулы~\eqref{eq:I_cyl} из раздела «Методика») и составили
$I_z^{(\text{th})} = 2{,}20\cdot10^{-3}$~кг·м$^2$ и
$I_x^{(\text{th})} = 1{,}56\cdot10^{-3}$~кг·м$^2$.

Периоды крутильных колебаний системы «рамка + цилиндр» были измерены для
двух взаимно перпендикулярных осей (табл.~\ref{tab:cyl-T}). Для оси $Oz$
получено среднее значение $T_z = 3{,}17$~с, для боковой оси $Ox$ —
$T_x = 3{,}00$~с. Используя формулу связи периода крутильных колебаний
с моментом инерции (формула~\eqref{eq:I_T}), из периода $T_z$ и известного
теоретического значения $I_z^{(\text{th})}$ был определён модуль кручения
подвесной проволоки $f$. Далее, подставляя тот же модуль в формулу для
периода $T_x$, были вычислены экспериментальные моменты инерции цилиндра
относительно обеих осей. Полученные значения
$I_z^{(\text{exp})}$ и $I_x^{(\text{exp})}$ отличаются от
$I_z^{(\text{th})}$ и $I_x^{(\text{th})}$ менее чем на один процент,
причём относительное расхождение для оси $Ox$ составляет порядка
$0{,}4\,\%$, что существенно меньше суммарной погрешности измерений
геометрических размеров и периодов.

По найденным главным моментам инерции был построен эллипсоид инерции
цилиндра. Для наглядности рассмотрено его сечение плоскостью $OXOZ$
(рис.~\ref{fig:sec_cyl}), где по горизонтальной оси откладываются
координаты, пропорциональные величине $1/\sqrt{I_x}$, а по вертикальной —
$1/\sqrt{I_z}$. Красные точки соответствуют пересечениям эллипса с осями,
то есть направлениям, вдоль которых измерялись периоды колебаний.
Видно, что эллипс вытянут вдоль оси $OX$ и сжат вдоль $OZ$, что отражает
больший момент инерции при вращении вокруг оси симметрии и меньший —
при вращении вокруг диаметра основания. Таким образом, сечение эллипсоида
инерции качественно и количественно согласуется с результатами
эксперимента и теорией \cite{Sivukhin}.

\begin{figure}[h!]
\centering
\includegraphics[width=0.6\textwidth]{section_cylinder_exp_final.png}
\caption{Сечение эллипсоида инерции цилиндра плоскостью $OXOZ$.}
\label{fig:sec_cyl}
\end{figure}

\subsection*{Прямоугольный параллелепипед}

Для прямоугольного параллелепипеда были измерены периоды крутильных
колебаний системы относительно трёх осей, направленных вдоль его рёбер
$Ox$, $Oy$, $Oz$, а также относительно диагонали, соединяющей середины
противоположных рёбер (ось $MM'$), и осей, проходящих через середины
рёбер и диагональ параллелепипеда (оси $EE'$, $PP'$ и $AC'$;
см. табл.~3 и~4 отчёта). Эти периоды использовались для проверки
соотношений между моментами инерции, следующих из уравнения эллипсоида
инерции параллелепипеда (формулы типа~(10)–(11) в разделе «Методика»,
выведенные из общей зависимости момента инерции от направления оси
вращения).

Подстановка экспериментальных периодов в указанные соотношения показала,
что линейные комбинации $T^2$ для диагональных осей (EE', PP', AC')
согласуются с комбинациями для осей $Ox$, $Oy$, $Oz$ в пределах
экспериментальной погрешности. Это означает, что зависимость момента
инерции параллелепипеда от направления оси вращения действительно имеет
эллипсоидальный характер.

Форма эллипсоида инерции параллелепипеда наглядно иллюстрируется
его сечением плоскостью $AA'M$ (совпадающей с плоскостью $OXOZ$),
построенным по экспериментальным данным.
Красные точки соответствуют направлениям главных и дополнительных осей,
для которых измерялись периоды крутильных колебаний. Эллипс заметно
асимметричен, что отражает различие моментов инерции вдоль длинной и
короткой сторон параллелепипеда. При этом все экспериментальные точки
лежат на общей эллиптической кривой в пределах погрешностей, что
подтверждает корректность модели эллипсоида инерции для данного тела.

\begin{figure}[H]
\centering
\includegraphics[width=0.6\textwidth]{section_parallelepiped.png}
\caption{Сечение эллипсоида инерции прямоугольного параллелепипеда
плоскостью $AA'M$ ($OXOZ$).}
\label{fig:sec_par}
\end{figure}

\subsection*{Куб}

По результатам измерений длины ребра куба (табл.~\ref{tab:cube-geom})
получено значение $a = 92{,}7$~мм. Периоды крутильных колебаний системы
«рамка + куб» для трёх взаимно перпендикулярных осей
$AA'$, $MM'$ и $DD'$, проходящих через центр масс, оказались практически
одинаковыми: $T_{AA'} \approx 3{,}02$~с, $T_{MM'} \approx 3{,}03$~с и
$T_{DD'} \approx 3{,}03$~с. Максимальное относительное отклонение
от среднего периода не превышает $0{,}3\,\%$, а соответствующие моменты
инерции, рассчитанные по формуле~\eqref{eq:I_T}, совпадают в пределах
экспериментальной ошибки.

Это означает, что все главные моменты инерции куба практически равны,
что и должно иметь место для тела с кубической симметрией. Геометрически
этому соответствует эллипсоид инерции, вырождающийся в сферу. На
рис.~\ref{fig:sec_cube} показано сечение этого эллипсоида плоскостью
$OXOZ$, построенное по экспериментальным данным. Эллипс на графике
имеет вид почти идеального круга; небольшие отклонения от круговой формы
обусловлены случайными погрешностями измерений периодов.

\begin{figure}[H]
\centering
\includegraphics[width=0.6\textwidth]{section_cube_exp_final.png}
\caption{Сечение эллипсоида инерции куба плоскостью $OXOZ$
(практически круг).}
\label{fig:sec_cube}
\end{figure}


Проведённые измерения показали, что для цилиндра экспериментальные
моменты инерции как по абсолютным значениям, так и по их отношению,
хорошо согласуются с теоретическими выражениями для однородного цилиндра
(расхождения не превышают одного процента). Для прямоугольного
параллелепипеда проверены соотношения между периодами крутильных
колебаний относительно различных осей, вытекающие из уравнения
эллипсоида инерции; все они выполняются в пределах экспериментальных
погрешностей. Для куба установлено практически полное равенство моментов
инерции относительно трёх взаимно перпендикулярных осей, проходящих
через центр масс, что соответствует вырождению эллипсоида инерции в
сферу.

Таким образом, экспериментальные данные находятся в полном согласии
с теорией эллипсоида инерции твёрдых тел и показывают,
что метод крутильных колебаний позволяет надёжно определять инерционные
характеристики реальных тел и визуализировать их в виде эллипсоидов
(или сфер) инерции.

\newpage
\section*{Выводы}

В работе экспериментально исследованы крутильные колебания системы
«рамка + твёрдое тело» и на этой основе построены эллипсоиды инерции
нескольких тел простой формы. Собственный период пустой рамки,
использовавшийся при всех расчётах моментов инерции, составил
$T_0 = 2{,}57$~с.

Для сплошного цилиндра по измеренным
периодам крутильных колебаний получены экспериментальные моменты инерции
относительно оси симметрии и диаметра основания:
$I_z^{(\text{exp})} \approx 2{,}20\cdot10^{-3}$~кг·м$^2$,
$I_x^{(\text{exp})} \approx 1{,}55\cdot10^{-3}$~кг·м$^2$.
Эти значения хорошо согласуются с теоретическими
$I_z^{(\text{th})} = 2{,}20\cdot10^{-3}$~кг·м$^2$ и
$I_x^{(\text{th})} = 1{,}56\cdot10^{-3}$~кг·м$^2$; относительное
расхождение не превышает $\sim 1\%$. Одновременно был оценён модуль
кручения подвесной проволоки $f \approx 2{,}5\cdot10^{-2}$~Н·м.
Сечение эллипсоида инерции цилиндра в плоскости $OXOZ$ (рис.~\ref{fig:sec_cyl})
имеет вид вытянутого эллипса, что наглядно отражает больший момент
инерции при вращении вокруг оси симметрии по сравнению с вращением
вокруг диаметра основания.

Для прямоугольного параллелепипеда измерены периоды крутильных
колебаний относительно осей, направленных вдоль рёбер ($T_x$, $T_y$,
$T_z$), и относительно осей, проходящих через середины рёбер и диагональ
тела ($T_{EE'}$, $T_{PP'}$, $T_{AC'}$, $T_{MM'}$).
Подстановка экспериментальных значений в соотношения, вытекающие
из уравнения эллипсоида инерции, показала выполнение этих соотношений
в пределах погрешности измерений. Сечение эллипсоида инерции
параллелепипеда (рис.~\ref{fig:sec_par}) представляет собой заметно
неравноосный эллипс, что соответствует различию моментов инерции вдоль
разных рёбер и подтверждает эллипсоидальный характер зависимости
момента инерции от направления оси вращения.

Для куба с ребром $a = 92{,}7$~мм получены периоды крутильных колебаний
относительно трёх взаимно перпендикулярных осей, проходящих через центр
массы: $T_{AA'} \approx 3{,}02$~с, $T_{MM'} \approx 3{,}03$~с,
$T_{DD'} \approx 3{,}03$~с. Разброс не превышает $0{,}3\%$, а соответствующие
моменты инерции совпадают в пределах ошибки, что указывает на равенство
главных моментов инерции куба. Сечение эллипсоида инерции в плоскости
$OXOZ$ (рис.~\ref{fig:sec_cube}) близко к окружности, т.е. эллипсоид
практически вырождается в сферу, как и предсказывает теория для тела с
кубической симметрией.

Полученные результаты демонстрируют, что метод крутильных колебаний
позволяет не только измерять отдельные моменты инерции, но и фактически
восстанавливать форму эллипсоида инерции твёрдого тела. Это напрямую
связывает проведённый лабораторный эксперимент с задачами, описанными
во введении: для реальных приборов, например, оптико-электронного блока
гиростабилизированной платформы беспилотного летательного аппарата,
знание главных моментов инерции и ориентации главных осей критически
важно для корректного расчёта динамики и настройки системы управления.
Освоенный в работе экспериментальный подход может быть использован для
таких сложных объектов после их окончательной сборки, обеспечивая
повышение точности моделирования и надёжности работы реальных
технических систем.

\section*{Приложение}
Таблица 1: Колебания пустой рамки.
В таблице приведены измеренное время t (для фиксированного числа колебаний)
и соответствующий период T одной крутильной колебательной рамки без тела.

\begin{table}[H]
\centering
\caption{Колебания рамки}
\label{tab:frame}
\begin{tabular}{|c|c|c|c|c|c|}
\hline
$t$, c & 38{,}4 & 38{,}6 & 38{,}6 & 38{,}6 & 38{,}5 \\ \hline
$T$, c & 2{,}56 & 2{,}57 & 2{,}57 & 2{,}57 & 2{,}57 \\ \hline
\end{tabular}
\end{table}

Таблица 2: Геометрические размеры цилиндра.
h — высота цилиндра, d — диаметр основания.
Приведены 10 измерений каждого параметра, их средние значения и
оценённые среднеквадратичные погрешности $\sigma$.

\begin{table}[H]
\centering
\caption{Измерения размеров цилиндра}
\label{tab:cyl-geom}
\begin{tabular}{|c|cccccccccc|c|c|}
\hline
№       & 1    & 2    & 3    & 4    & 5    & 6    & 7    & 8    & 9    & 10   & Среднее & $\sigma$ \\ \hline
$h$, мм & 49{,}1 & 49{,}5 & 49{,}3 & 49{,}3 & 49{,}2 & 49{,}2 & 49{,}2 & 49{,}4 & 49{,}2 & 49{,}2 & 49{,}3 & 0{,}2 \\
$d$, мм & 88{,}3 & 88{,}1 & 88{,}1 & 88{,}1 & 88{,}1 & 88{,}1 & 88{,}1 & 88{,}1 & 88{,}1 & 88{,}0 & 88{,}1 & 0{,}2 \\ \hline
\end{tabular}
\end{table}

Таблица 3: Периоды крутильных колебаний системы «рамка + цилиндр».
В верхней части — измерения при вращении вокруг главной оси цилиндра (оси симметрии),
в нижней — при вращении вокруг боковой оси (диаметра основания).
t — время нескольких колебаний, T — соответствующий период одного колебания.

\begin{table}[H]
\centering
\caption{Колебания цилиндра}
\label{tab:cyl-T}
\begin{tabular}{|c|ccccc|}
\hline
\multicolumn{6}{|c|}{Главная ось} \\ \hline
$t$, c & 47{,}9 & 47{,}6 & 47{,}2 & 47{,}4 & 47{,}6 \\ \hline
$T$, c & 3{,}19 & 3{,}17 & 3{,}15 & 3{,}16 & 3{,}17 \\ \hline
\multicolumn{6}{|c|}{Боковая ось} \\ \hline
$t$, c & 45{,}1 & 45{,}0 & 45{,}3 & 45{,}0 & 44{,}9 \\ \hline
$T$, c & 3{,}01 & 3{,}00 & 3{,}02 & 3{,}00 & 2{,}99 \\ \hline
\end{tabular}
\end{table}

Таблица 4: Измерения для куба.
В первой строке приведены 10 измерений длины ребра a куба и их статистическая обработка.
Далее даны времена t и периоды T крутильных колебаний для трёх осей,
проходящих через центр куба: AA' (по ребру), MM' и DD'.
В последнем столбце каждой подтаблицы указано среднее значение периода.

\begin{table}[H]
\centering
\caption{Измерения для куба}
\label{tab:cube-geom}
\begin{tabular}{|c|cccccccccc|c|c|}
\hline
\multicolumn{13}{|c|}{Измерения длины} \\ \hline
№      & 1    & 2    & 3    & 4    & 5    & 6    & 7    & 8    & 9    & 10   & $\bar a$ \\ \hline
\end{tabular}
\end{table}

\begin{table}[H]
\centering
\begin{tabular}{|c|ccccc|c|}
\hline
\multicolumn{7}{|c|}{Ось $MM'$} \\ \hline
№    & 1    & 2    & 3    & 4    & 5    & Среднее \\ \hline
$t$, c & 45{,}6 & 45{,}6 & 45{,}5 & 45{,}3 & 45{,}5 & 45{,}5 \\ 
$T$, c & 3{,}04 & 3{,}04 & 3{,}03 & 3{,}02 & 3{,}03 & 3{,}03 \\ \hline
\end{tabular}
\end{table}

\begin{table}[H]
\centering
\begin{tabular}{|c|ccccc|c|}
\hline
\multicolumn{7}{|c|}{Ось $DD'$} \\ \hline
№    & 1    & 2    & 3    & 4    & 5    & Среднее \\ \hline
$t$, c & 45{,}5 & 45{,}7 & 45{,}4 & 45{,}5 & 45{,}5 & 45{,}5 \\ 
$T$, c & 3{,}03 & 3{,}05 & 3{,}03 & 3{,}03 & 3{,}03 & 3{,}03 \\ \hline
\end{tabular}
\end{table}

\newpage


\begin{thebibliography}{99}

\bibitem{Landau}
Л.\,Д. Ландау, Е.\,М. Лифшиц, \textit{Механика}. — М.: Наука, 1988. — 216~с.

\bibitem{Saveliev}
И.\,В. Савельев, \textit{Курс общей физики. Том II. Электричество и колебания}. — М.: Наука, 1989. — 560~с.

\bibitem{Gladun}
Д.\,А. Гладун, \textit{Лабораторный практикум по общей физике}. — М.: Издательство МФТИ, 2012. — 312~с.

\end{thebibliography}
\end{document}
